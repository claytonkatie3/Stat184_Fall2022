% Options for packages loaded elsewhere
\PassOptionsToPackage{unicode}{hyperref}
\PassOptionsToPackage{hyphens}{url}
%
\documentclass[
]{article}
\usepackage{amsmath,amssymb}
\usepackage{lmodern}
\usepackage{iftex}
\ifPDFTeX
  \usepackage[T1]{fontenc}
  \usepackage[utf8]{inputenc}
  \usepackage{textcomp} % provide euro and other symbols
\else % if luatex or xetex
  \usepackage{unicode-math}
  \defaultfontfeatures{Scale=MatchLowercase}
  \defaultfontfeatures[\rmfamily]{Ligatures=TeX,Scale=1}
\fi
% Use upquote if available, for straight quotes in verbatim environments
\IfFileExists{upquote.sty}{\usepackage{upquote}}{}
\IfFileExists{microtype.sty}{% use microtype if available
  \usepackage[]{microtype}
  \UseMicrotypeSet[protrusion]{basicmath} % disable protrusion for tt fonts
}{}
\makeatletter
\@ifundefined{KOMAClassName}{% if non-KOMA class
  \IfFileExists{parskip.sty}{%
    \usepackage{parskip}
  }{% else
    \setlength{\parindent}{0pt}
    \setlength{\parskip}{6pt plus 2pt minus 1pt}}
}{% if KOMA class
  \KOMAoptions{parskip=half}}
\makeatother
\usepackage{xcolor}
\usepackage[margin=1in]{geometry}
\usepackage{color}
\usepackage{fancyvrb}
\newcommand{\VerbBar}{|}
\newcommand{\VERB}{\Verb[commandchars=\\\{\}]}
\DefineVerbatimEnvironment{Highlighting}{Verbatim}{commandchars=\\\{\}}
% Add ',fontsize=\small' for more characters per line
\usepackage{framed}
\definecolor{shadecolor}{RGB}{248,248,248}
\newenvironment{Shaded}{\begin{snugshade}}{\end{snugshade}}
\newcommand{\AlertTok}[1]{\textcolor[rgb]{0.94,0.16,0.16}{#1}}
\newcommand{\AnnotationTok}[1]{\textcolor[rgb]{0.56,0.35,0.01}{\textbf{\textit{#1}}}}
\newcommand{\AttributeTok}[1]{\textcolor[rgb]{0.77,0.63,0.00}{#1}}
\newcommand{\BaseNTok}[1]{\textcolor[rgb]{0.00,0.00,0.81}{#1}}
\newcommand{\BuiltInTok}[1]{#1}
\newcommand{\CharTok}[1]{\textcolor[rgb]{0.31,0.60,0.02}{#1}}
\newcommand{\CommentTok}[1]{\textcolor[rgb]{0.56,0.35,0.01}{\textit{#1}}}
\newcommand{\CommentVarTok}[1]{\textcolor[rgb]{0.56,0.35,0.01}{\textbf{\textit{#1}}}}
\newcommand{\ConstantTok}[1]{\textcolor[rgb]{0.00,0.00,0.00}{#1}}
\newcommand{\ControlFlowTok}[1]{\textcolor[rgb]{0.13,0.29,0.53}{\textbf{#1}}}
\newcommand{\DataTypeTok}[1]{\textcolor[rgb]{0.13,0.29,0.53}{#1}}
\newcommand{\DecValTok}[1]{\textcolor[rgb]{0.00,0.00,0.81}{#1}}
\newcommand{\DocumentationTok}[1]{\textcolor[rgb]{0.56,0.35,0.01}{\textbf{\textit{#1}}}}
\newcommand{\ErrorTok}[1]{\textcolor[rgb]{0.64,0.00,0.00}{\textbf{#1}}}
\newcommand{\ExtensionTok}[1]{#1}
\newcommand{\FloatTok}[1]{\textcolor[rgb]{0.00,0.00,0.81}{#1}}
\newcommand{\FunctionTok}[1]{\textcolor[rgb]{0.00,0.00,0.00}{#1}}
\newcommand{\ImportTok}[1]{#1}
\newcommand{\InformationTok}[1]{\textcolor[rgb]{0.56,0.35,0.01}{\textbf{\textit{#1}}}}
\newcommand{\KeywordTok}[1]{\textcolor[rgb]{0.13,0.29,0.53}{\textbf{#1}}}
\newcommand{\NormalTok}[1]{#1}
\newcommand{\OperatorTok}[1]{\textcolor[rgb]{0.81,0.36,0.00}{\textbf{#1}}}
\newcommand{\OtherTok}[1]{\textcolor[rgb]{0.56,0.35,0.01}{#1}}
\newcommand{\PreprocessorTok}[1]{\textcolor[rgb]{0.56,0.35,0.01}{\textit{#1}}}
\newcommand{\RegionMarkerTok}[1]{#1}
\newcommand{\SpecialCharTok}[1]{\textcolor[rgb]{0.00,0.00,0.00}{#1}}
\newcommand{\SpecialStringTok}[1]{\textcolor[rgb]{0.31,0.60,0.02}{#1}}
\newcommand{\StringTok}[1]{\textcolor[rgb]{0.31,0.60,0.02}{#1}}
\newcommand{\VariableTok}[1]{\textcolor[rgb]{0.00,0.00,0.00}{#1}}
\newcommand{\VerbatimStringTok}[1]{\textcolor[rgb]{0.31,0.60,0.02}{#1}}
\newcommand{\WarningTok}[1]{\textcolor[rgb]{0.56,0.35,0.01}{\textbf{\textit{#1}}}}
\usepackage{graphicx}
\makeatletter
\def\maxwidth{\ifdim\Gin@nat@width>\linewidth\linewidth\else\Gin@nat@width\fi}
\def\maxheight{\ifdim\Gin@nat@height>\textheight\textheight\else\Gin@nat@height\fi}
\makeatother
% Scale images if necessary, so that they will not overflow the page
% margins by default, and it is still possible to overwrite the defaults
% using explicit options in \includegraphics[width, height, ...]{}
\setkeys{Gin}{width=\maxwidth,height=\maxheight,keepaspectratio}
% Set default figure placement to htbp
\makeatletter
\def\fps@figure{htbp}
\makeatother
\setlength{\emergencystretch}{3em} % prevent overfull lines
\providecommand{\tightlist}{%
  \setlength{\itemsep}{0pt}\setlength{\parskip}{0pt}}
\setcounter{secnumdepth}{-\maxdimen} % remove section numbering
\usepackage{booktabs}
\usepackage{longtable}
\usepackage{array}
\usepackage{multirow}
\usepackage{wrapfig}
\usepackage{float}
\usepackage{colortbl}
\usepackage{pdflscape}
\usepackage{tabu}
\usepackage{threeparttable}
\usepackage{threeparttablex}
\usepackage[normalem]{ulem}
\usepackage{makecell}
\usepackage{xcolor}
\ifLuaTeX
  \usepackage{selnolig}  % disable illegal ligatures
\fi
\IfFileExists{bookmark.sty}{\usepackage{bookmark}}{\usepackage{hyperref}}
\IfFileExists{xurl.sty}{\usepackage{xurl}}{} % add URL line breaks if available
\urlstyle{same} % disable monospaced font for URLs
\hypersetup{
  pdftitle={Activity 8},
  pdfauthor={Katie Clayton},
  hidelinks,
  pdfcreator={LaTeX via pandoc}}

\title{Activity 8}
\author{Katie Clayton}
\date{2022-11-02}

\begin{document}
\maketitle

\hypertarget{r-markdown}{%
\subsection{R Markdown}\label{r-markdown}}

This is an R Markdown document. Markdown is a simple formatting syntax
for authoring HTML, PDF, and MS Word documents. For more details on
using R Markdown see \url{http://rmarkdown.rstudio.com}.

When you click the \textbf{Knit} button a document will be generated
that includes both content as well as the output of any embedded R code
chunks within the document. You can embed an R code chunk like this:

\begin{Shaded}
\begin{Highlighting}[]
\FunctionTok{summary}\NormalTok{(cars)}
\end{Highlighting}
\end{Shaded}

\begin{verbatim}
##      speed           dist       
##  Min.   : 4.0   Min.   :  2.00  
##  1st Qu.:12.0   1st Qu.: 26.00  
##  Median :15.0   Median : 36.00  
##  Mean   :15.4   Mean   : 42.98  
##  3rd Qu.:19.0   3rd Qu.: 56.00  
##  Max.   :25.0   Max.   :120.00
\end{verbatim}

\hypertarget{collatz-conjecture}{%
\section{\texorpdfstring{\textbf{Collatz
Conjecture}}{Collatz Conjecture}}\label{collatz-conjecture}}

The collatz conjecture is used to transform every positive integer into
one. Two arithmetic equations are input using if statements and else if
statements in order to get to the integer one. If the number is even,
the function will use the number and divide it by two. If the number is
odd, it will be multiplied by three and added to one. Once the number
gets to one, the function will stop. The purpose is to find out the most
frequent stopping time based off of this function.

\begin{Shaded}
\begin{Highlighting}[]
\CommentTok{\#name function and add nouns}
\NormalTok{getCollatz }\OtherTok{\textless{}{-}} \ControlFlowTok{function}\NormalTok{(number,}\AttributeTok{count=}\DecValTok{0}\NormalTok{)}
  \CommentTok{\#first part of collatz conjecture, when number is 1}
\NormalTok{\{}\ControlFlowTok{if}\NormalTok{ (number }\SpecialCharTok{==} \DecValTok{1}\NormalTok{)\{}
  \FunctionTok{return}\NormalTok{ (count)\}}
  \CommentTok{\#when number is even, divide by two}
  \ControlFlowTok{else} \ControlFlowTok{if}\NormalTok{ (number}\SpecialCharTok{\%\%}\DecValTok{2}\SpecialCharTok{==}\DecValTok{0}\NormalTok{)}
\NormalTok{    \{}\FunctionTok{getCollatz}\NormalTok{(number}\SpecialCharTok{/}\DecValTok{2}\NormalTok{,count}\SpecialCharTok{+}\DecValTok{1}\NormalTok{)\}}
  \CommentTok{\#when number is odd, multiply by 3 and add 1}
  \ControlFlowTok{else}\NormalTok{\{}
    \FunctionTok{getCollatz}\NormalTok{(number}\SpecialCharTok{*}\DecValTok{3}\SpecialCharTok{+}\DecValTok{1}\NormalTok{,count}\SpecialCharTok{+}\DecValTok{1}\NormalTok{)\}\}}
\CommentTok{\# create function for histogram}
\NormalTok{stoppingNumbers }\OtherTok{\textless{}{-}} \FunctionTok{sapply}\NormalTok{(}
\CommentTok{\# label x axis and incldue function that needs to be called}
\AttributeTok{X =} \DecValTok{1}\SpecialCharTok{:}\DecValTok{10000}\NormalTok{,}
\AttributeTok{FUN =}\NormalTok{ getCollatz}
\NormalTok{)}
\CommentTok{\# make the histogram}
\FunctionTok{hist}\NormalTok{(stoppingNumbers)}
\end{Highlighting}
\end{Shaded}

\includegraphics{activity-8_ClaytonK_files/figure-latex/unnamed-chunk-1-1.pdf}

According to the histogram, the most frequent stopping number is 50.
This means that the function, on average, runs through the arithmetic
operations 50 times before outputting the number one.

\hypertarget{diamonds-data-visualizations}{%
\section{Diamonds Data
Visualizations}\label{diamonds-data-visualizations}}

A viewer can gain the most knowledge from the diamonds data through a
data visualization. A scatter plot is most appropriate for this set of
data as it can include each observation and also a line of best fit to
show the linear regression as the weight of the diamond increases. A
summary table is also useful as it can display actual numbers and
percentages, making it easy for a viewer to compare differences between
each diamond.

\begin{Shaded}
\begin{Highlighting}[]
\CommentTok{\# filter data so there is no more than four groupings on one graph}
\FunctionTok{library}\NormalTok{(ggplot2)}
\FunctionTok{data}\NormalTok{(diamonds)}
\FunctionTok{library}\NormalTok{(dplyr)}
\end{Highlighting}
\end{Shaded}

\begin{verbatim}
## 
## Attaching package: 'dplyr'
\end{verbatim}

\begin{verbatim}
## The following objects are masked from 'package:stats':
## 
##     filter, lag
\end{verbatim}

\begin{verbatim}
## The following objects are masked from 'package:base':
## 
##     intersect, setdiff, setequal, union
\end{verbatim}

\begin{Shaded}
\begin{Highlighting}[]
\NormalTok{diamonds }\SpecialCharTok{\%\textgreater{}\%}
\FunctionTok{filter}\NormalTok{(color }\SpecialCharTok{\%in\%} \FunctionTok{c}\NormalTok{(}\StringTok{"J"}\NormalTok{, }\StringTok{"E"}\NormalTok{, }\StringTok{"I"}\NormalTok{)) }\SpecialCharTok{\%\textgreater{}\%}

\CommentTok{\# create graph using ggplot and assign variables}
\FunctionTok{ggplot}\NormalTok{(}
\AttributeTok{mapping =} \FunctionTok{aes}\NormalTok{(}\AttributeTok{x =}\NormalTok{ carat, }\AttributeTok{y =}\NormalTok{ price, }\AttributeTok{colour =}\NormalTok{ color)}
\NormalTok{) }\SpecialCharTok{+}

\CommentTok{\# establish size for the line of best fit and the points, as well as shape}
\CommentTok{\# for points}
\FunctionTok{geom\_point}\NormalTok{(}\AttributeTok{shape =} \StringTok{"circle"}\NormalTok{, }\AttributeTok{size =} \FloatTok{1.5}\NormalTok{) }\SpecialCharTok{+}
\FunctionTok{geom\_smooth}\NormalTok{(}\AttributeTok{span =} \FloatTok{0.75}\NormalTok{, }\AttributeTok{se =} \ConstantTok{FALSE}\NormalTok{) }\SpecialCharTok{+}

\CommentTok{\# give each color a color which will display the legend}
\FunctionTok{scale\_color\_manual}\NormalTok{(}
\AttributeTok{values =} \FunctionTok{c}\NormalTok{(}\AttributeTok{D =} \StringTok{"\#0D0B0C"}\NormalTok{,}
\AttributeTok{F =} \StringTok{"\#888386"}\NormalTok{,}
\AttributeTok{G =} \StringTok{"\#00A7FF"}\NormalTok{,}
\AttributeTok{H =} \StringTok{"\#FF8E00"}\NormalTok{,}
\AttributeTok{J =} \StringTok{"\#000000"}\NormalTok{,}
\AttributeTok{E =} \StringTok{"\#8A8591"}\NormalTok{,}
\AttributeTok{I =} \StringTok{"\#DC9423"}\NormalTok{)}
\NormalTok{) }\SpecialCharTok{+}

\CommentTok{\# label the axis, add a title}
\FunctionTok{labs}\NormalTok{(}
\AttributeTok{x =} \StringTok{"Weight of Diamond (carats)"}\NormalTok{,}
\AttributeTok{y =} \StringTok{"Price of Diamond (dollars)"}\NormalTok{,}
\AttributeTok{title =} \StringTok{"Price of Diamond (dollars) vs. Color and Weight (carats)"}
\NormalTok{)}\SpecialCharTok{+}
\FunctionTok{theme\_minimal}\NormalTok{()}
\end{Highlighting}
\end{Shaded}

\begin{verbatim}
## `geom_smooth()` using method = 'gam' and formula 'y ~ s(x, bs = "cs")'
\end{verbatim}

\includegraphics{activity-8_ClaytonK_files/figure-latex/unnamed-chunk-2-1.pdf}

\begin{Shaded}
\begin{Highlighting}[]
\CommentTok{\# create second graph with the remaining four variables}
\FunctionTok{library}\NormalTok{(ggplot2)}
\FunctionTok{data}\NormalTok{(diamonds)}
\FunctionTok{library}\NormalTok{(dplyr)}
\NormalTok{diamonds }\SpecialCharTok{\%\textgreater{}\%}
\FunctionTok{filter}\NormalTok{(color }\SpecialCharTok{\%in\%} \FunctionTok{c}\NormalTok{(}\StringTok{"D"}\NormalTok{, }\StringTok{"F"}\NormalTok{, }\StringTok{"G"}\NormalTok{, }\StringTok{"H"}\NormalTok{)) }\SpecialCharTok{\%\textgreater{}\%}

\CommentTok{\# create graph using ggplot and assign variables}
\FunctionTok{ggplot}\NormalTok{(}
\AttributeTok{mapping =} \FunctionTok{aes}\NormalTok{(}\AttributeTok{x =}\NormalTok{ carat, }\AttributeTok{y =}\NormalTok{ price, }\AttributeTok{colour =}\NormalTok{ color)}
\NormalTok{) }\SpecialCharTok{+}

\CommentTok{\# establish size for the line of best fit and the points, as well as shape}
\CommentTok{\# for points}
\FunctionTok{geom\_point}\NormalTok{(}\AttributeTok{shape =} \StringTok{"circle"}\NormalTok{, }\AttributeTok{size =} \FloatTok{1.5}\NormalTok{) }\SpecialCharTok{+}
\FunctionTok{geom\_smooth}\NormalTok{(}\AttributeTok{span =} \FloatTok{0.75}\NormalTok{, }\AttributeTok{se =} \ConstantTok{FALSE}\NormalTok{) }\SpecialCharTok{+}

\CommentTok{\# give each color a color which will display the legend}
\FunctionTok{scale\_color\_manual}\NormalTok{(}
\AttributeTok{values =} \FunctionTok{c}\NormalTok{(}\AttributeTok{D =} \StringTok{"\#0D0B0C"}\NormalTok{,}
\AttributeTok{F =} \StringTok{"\#888386"}\NormalTok{,}
\AttributeTok{G =} \StringTok{"\#00A7FF"}\NormalTok{,}
\AttributeTok{H =} \StringTok{"\#FF8E00"}\NormalTok{,}
\AttributeTok{J =} \StringTok{"\#000000"}\NormalTok{,}
\AttributeTok{E =} \StringTok{"\#8A8591"}\NormalTok{,}
\AttributeTok{I =} \StringTok{"\#DC9423"}\NormalTok{)}
\NormalTok{) }\SpecialCharTok{+}

\CommentTok{\# label the axis, add a title}
\FunctionTok{labs}\NormalTok{(}
\AttributeTok{x =} \StringTok{"Weight of Diamond (carats)"}\NormalTok{,}
\AttributeTok{y =} \StringTok{"Price of Diamond (dollars)"}\NormalTok{,}
\AttributeTok{title =} \StringTok{"Price of Diamond (dollars) vs. Color and Weight (carats)"}
\NormalTok{)}\SpecialCharTok{+}
\FunctionTok{theme\_minimal}\NormalTok{()}
\end{Highlighting}
\end{Shaded}

\begin{verbatim}
## `geom_smooth()` using method = 'gam' and formula 'y ~ s(x, bs = "cs")'
\end{verbatim}

\includegraphics{activity-8_ClaytonK_files/figure-latex/unnamed-chunk-2-2.pdf}

As shown above, the different types of diamonds are listed along the
side of the graph. Each axis is specifically labeled and the lines of
best fit show the different trends between the different types of
diamonds. In general, a reader can see that the price of a diamond
increases as the weight of the diamond increases. However, looking at
the different graphs, there can be distinct differences as to how much
faster one diamond increases compared to another by looking at the
steepness of the line.

\begin{Shaded}
\begin{Highlighting}[]
\DocumentationTok{\#\# load necessary packages}
\FunctionTok{library}\NormalTok{(ggplot2)}
\FunctionTok{library}\NormalTok{(dplyr)}
\FunctionTok{library}\NormalTok{(knitr)}
\FunctionTok{library}\NormalTok{(kableExtra)}
\end{Highlighting}
\end{Shaded}

\begin{verbatim}
## 
## Attaching package: 'kableExtra'
\end{verbatim}

\begin{verbatim}
## The following object is masked from 'package:dplyr':
## 
##     group_rows
\end{verbatim}

\begin{Shaded}
\begin{Highlighting}[]
\FunctionTok{library}\NormalTok{(janitor)}
\end{Highlighting}
\end{Shaded}

\begin{verbatim}
## 
## Attaching package: 'janitor'
\end{verbatim}

\begin{verbatim}
## The following objects are masked from 'package:stats':
## 
##     chisq.test, fisher.test
\end{verbatim}

\begin{Shaded}
\begin{Highlighting}[]
\CommentTok{\# load diamonds data}
\FunctionTok{data}\NormalTok{(diamonds)}
\CommentTok{\# name variable}
\NormalTok{depthDiamonds }\OtherTok{\textless{}{-}}\NormalTok{ diamonds }\SpecialCharTok{\%\textgreater{}\%}
\CommentTok{\# choose cut and the depth which is z}
\FunctionTok{group\_by}\NormalTok{(cut) }\SpecialCharTok{\%\textgreater{}\%}
\FunctionTok{select}\NormalTok{(cut, z) }\SpecialCharTok{\%\textgreater{}\%}
\CommentTok{\# input the 10 statistics needed, not forgetting na.rm = TRUE at the end of the line}
\FunctionTok{summarize}\NormalTok{(}
\FunctionTok{across}\NormalTok{(}
\AttributeTok{.cols =} \FunctionTok{where}\NormalTok{(is.numeric),}
\AttributeTok{.fns =} \FunctionTok{list}\NormalTok{(}
\AttributeTok{min =} \SpecialCharTok{\textasciitilde{}}\FunctionTok{min}\NormalTok{(z, }\AttributeTok{na.rm =} \ConstantTok{TRUE}\NormalTok{),}
\AttributeTok{Q1 =} \SpecialCharTok{\textasciitilde{}}\FunctionTok{quantile}\NormalTok{(z, }\AttributeTok{probs =} \FloatTok{0.20}\NormalTok{, }\AttributeTok{na.rm =} \ConstantTok{TRUE}\NormalTok{),}
\AttributeTok{Q2 =} \SpecialCharTok{\textasciitilde{}}\FunctionTok{quantile}\NormalTok{(z, }\AttributeTok{probs =} \FloatTok{0.40}\NormalTok{, }\AttributeTok{na.rm =} \ConstantTok{TRUE}\NormalTok{),}
\AttributeTok{median =} \SpecialCharTok{\textasciitilde{}}\FunctionTok{median}\NormalTok{(z, }\AttributeTok{na.rm =} \ConstantTok{TRUE}\NormalTok{),}
\AttributeTok{Q3 =} \SpecialCharTok{\textasciitilde{}}\FunctionTok{quantile}\NormalTok{(z, }\AttributeTok{probs =} \FloatTok{0.60}\NormalTok{, }\AttributeTok{na.rm =} \ConstantTok{TRUE}\NormalTok{),}
\AttributeTok{Q4 =} \SpecialCharTok{\textasciitilde{}}\FunctionTok{quantile}\NormalTok{(z, }\AttributeTok{probs =} \FloatTok{0.80}\NormalTok{, }\AttributeTok{na.rm =} \ConstantTok{TRUE}\NormalTok{),}
\AttributeTok{max =} \SpecialCharTok{\textasciitilde{}}\FunctionTok{max}\NormalTok{(z, }\AttributeTok{na.rm =} \ConstantTok{TRUE}\NormalTok{),}
\AttributeTok{sam =} \SpecialCharTok{\textasciitilde{}}\FunctionTok{mean}\NormalTok{(z, }\AttributeTok{na.rm =} \ConstantTok{TRUE}\NormalTok{),}
\AttributeTok{sasd =} \SpecialCharTok{\textasciitilde{}}\FunctionTok{sd}\NormalTok{(z, }\AttributeTok{na.rm =} \ConstantTok{TRUE}\NormalTok{)}
\NormalTok{),}
\FunctionTok{round}\NormalTok{(}\AttributeTok{digits =} \DecValTok{2}\NormalTok{)}
\NormalTok{),}
\CommentTok{\# format the numbers so the big ones have a , if needed}
\AttributeTok{count =} \FunctionTok{format}\NormalTok{(}\FunctionTok{n}\NormalTok{(), }\AttributeTok{big.mark =} \StringTok{","}\NormalTok{),}
\NormalTok{)}
\CommentTok{\# give the columns names}
\FunctionTok{colnames}\NormalTok{(depthDiamonds) }\OtherTok{\textless{}{-}} \FunctionTok{c}\NormalTok{(}\StringTok{"Cut"}\NormalTok{, }\StringTok{"Min"}\NormalTok{, }\StringTok{"1st Quintile"}\NormalTok{, }\StringTok{"2nd Quintile"}\NormalTok{, }\StringTok{"Median"}\NormalTok{, }\StringTok{"3rd Quintile"}\NormalTok{, }\StringTok{"4th Quintile"}\NormalTok{, }\StringTok{"Max"}\NormalTok{, }\StringTok{"Arithmetic Mean"}\NormalTok{, }\StringTok{"Arithmetic Standard Deviation"}\NormalTok{, }\StringTok{"Count"}\NormalTok{)}
\CommentTok{\# use piping and kable function to output the table}
\NormalTok{depthDiamonds }\SpecialCharTok{\%\textgreater{}\%}
\FunctionTok{kable}\NormalTok{(}
\CommentTok{\# add title, grid lines and align the numbers with words}
\AttributeTok{caption =} \StringTok{"Statistics of the Depth vs Cut of a Diamond"}\NormalTok{,}
\AttributeTok{booktabs =} \ConstantTok{TRUE}\NormalTok{,}
\AttributeTok{align =} \FunctionTok{c}\NormalTok{(}\StringTok{"l"}\NormalTok{, }\FunctionTok{rep}\NormalTok{(}\StringTok{"c"}\NormalTok{, }\DecValTok{6}\NormalTok{))}
\NormalTok{) }\SpecialCharTok{\%\textgreater{}\%}
\NormalTok{kableExtra}\SpecialCharTok{::}\FunctionTok{kable\_styling}\NormalTok{(}
\AttributeTok{bootstrap\_options =} \FunctionTok{c}\NormalTok{(}\StringTok{"striped"}\NormalTok{, }\StringTok{"condensed"}\NormalTok{),}
\AttributeTok{font\_size =} \DecValTok{16}
\NormalTok{) }\SpecialCharTok{\%\textgreater{}\%}
\NormalTok{kableExtra}\SpecialCharTok{::}\FunctionTok{kable\_classic}\NormalTok{()}
\end{Highlighting}
\end{Shaded}

\begin{table}

\caption{\label{tab:Diamonds Data Summary Table}Statistics of the Depth vs Cut of a Diamond}
\centering
\fontsize{16}{18}\selectfont
\begin{tabular}[t]{lcccccclccc}
\toprule
Cut & Min & 1st Quintile & 2nd Quintile & Median & 3rd Quintile & 4th Quintile & Max & Arithmetic Mean & Arithmetic Standard Deviation & Count\\
\midrule
Fair & 0 & 3.47 & 3.836 & 3.97 & 4.07 & 4.49 & 6.98 & 3.982770 & 0.6516384 & 1,610\\
Good & 0 & 2.99 & 3.480 & 3.70 & 3.87 & 4.08 & 5.79 & 3.639507 & 0.6548925 & 4,906\\
Very Good & 0 & 2.83 & 3.320 & 3.56 & 3.82 & 4.10 & 31.80 & 3.559801 & 0.7302281 & 12,082\\
Premium & 0 & 2.85 & 3.440 & 3.72 & 3.94 & 4.26 & 8.06 & 3.647124 & 0.7311610 & 13,791\\
Ideal & 0 & 2.75 & 3.060 & 3.23 & 3.52 & 4.05 & 6.03 & 3.401448 & 0.6576481 & 21,551\\
\bottomrule
\end{tabular}
\end{table}

The summary table of the diamonds data includes each cut of diamond and
10 different statistics pertaining to the depth of the specific cut. A
viewer can gain a sense of how many diamonds there are per cut, which
type of cut has the highest maximum and much more. This summary table
would be most useful when trying to gain specific statistics about the
depth of a certain cut of a diamond.

\hypertarget{knowledge-from-stat-184}{%
\section{Knowledge from STAT 184}\label{knowledge-from-stat-184}}

Stat 184 has taught me a lot of different things so far. Coming into
this class, I had very little prior coding experience so I was very
nervous. Although, as the class has continued, I have become more
comfortable with reading code and typing it. I have also really enjoyed
using the different packages within R, especially ggplot. I have also
gained other knowledge about data visualizations from Kosslyn and Tufte.
They specified things to avoid, like ducks, as well as things to include
that will help make the visualization most pleasing to the eye. It has
been very helpful to have all the slides and other material posted in
order to look back on when getting confused. I have learned the
importance of using tidy data and also how to wrangle the data so it is
easier to work with. The lesson that I remember most and will continue
to apply is the idea of PCIP. By following those steps, it has led me to
solutions quicker. Before taking this class, I had zero experience in
statistical coding and now I feel as if I have tremendously increased my
knowledge on this topic.

\end{document}
